\phantomsection
\addcontentsline{toc}{chapter}{Reference}

\begin{Reference}
%%Introduction reference (14+7 21 reference)
%%LR Section2.1 (19 reference)
%%LR Section2.2 (14 reference)
%%LR Section2.3 (13 reference) 目前总共67个
%%LR Section2.4 ()
%%LR Section2.5 ()
\begin{flushleft}
Alam, U., Asghar, O., Azmi, S., and Malik, R. A. (2014). `General aspects of diabetes mellitus'. \textit{Handbook of clinical neurology}, 126, pp.211-222.
\end{flushleft}
\vspace{2pt}


\begin{flushleft}
Albert, D. P., Gesler, W. M., and Levergood, B. (eds.) (2000). \textit{Spatial analysis, GIS, and remote sensing applications in the health sciences}. Chelsea, MI: Ann Arbor Press.
\end{flushleft}
\vspace{2pt}


\begin{flushleft}
Anselin, L. (2005). `Exploring spatial data with GeoDaTM: a workbook'. \textit{Centre for Spatially Integrated Social Science}.
\end{flushleft}
\vspace{2pt}


\begin{flushleft}
Anselin, L., and Bera, A. K. (1998). `Spatial dependence in linear regression models with an introduction to spatial econometrics'. \textit{Statistics textbooks and monographs}, 155, pp.237-290.
\end{flushleft}
\vspace{2pt}


\begin{flushleft}
Artiga, S., and Hinton, E. (2018). `Beyond health care: the role of social determinants in promoting health and health equity'. \textit{Kaiser Family Foundation}, 10.
\end{flushleft}
\vspace{2pt}


\begin{flushleft}
Askari, M. H., Gupta, K., and Bengal, W. (2016). `Conceptualising medical geography'. \textit{Transactions}, 38(1), p.127.
\end{flushleft}
\vspace{2pt}


\begin{flushleft}
Baker, C. (2024). \textit{Constituency data: health conditions}. UK Parliament, House of Commons Library. Available at: \url{https://commonslibrary.parliament.uk/constituency-data-how-healthy-is-your-area/} (Accessed: 21 July 2024).
\end{flushleft}
\vspace{7pt}


\begin{flushleft}
Baker, C. (2021). \textit{Local health conditions prevalence estimates based on QOF}. GitHub repository. Available at: \url{https://github.com/houseofcommonslibrary/local-health-data-from-QOF} (Accessed: 15 July 2024).
\end{flushleft}
\vspace{7pt}


\begin{flushleft}
Berkman, L. F., Kawachi, I., and Glymour, M. M. (eds.) (2014). \textit{Social epidemiology}. Oxford University Press.
\end{flushleft}
\vspace{2pt}


\begin{flushleft}
Bilal, U., Auchincloss, A. H., and Diez-Roux, A. V. (2018). `Neighborhood environments and diabetes risk and control'. \textit{Current diabetes reports}, 18, pp.1-10.
\end{flushleft}
\vspace{2pt}


\begin{flushleft}
Boehme, A. K., Esenwa, C., and Elkind, M. S. (2017). `Stroke risk factors, genetics, and prevention'. \textit{Circulation research}, 120(3), pp.472-495.
\end{flushleft}
\vspace{2pt}


\begin{flushleft}
Borgnakke, W. S. (2016). `Modifiable risk factors for periodontitis and diabetes'. \textit{Current Oral Health Reports}, 3, pp.254-269.
\end{flushleft}
\vspace{2pt}


\begin{flushleft}
Borgnakke, W. S. (2016). `Non-modifiable risk factors for periodontitis and diabetes'. \textit{Current oral health reports}, 3, pp.270-281.
\end{flushleft}
\vspace{2pt}


\begin{flushleft}
Braveman, P. (2014). `What are health disparities and health equity? We need to be clear'. \textit{Public health reports}, 129(1\_suppl2), pp.5-8.
\end{flushleft}
\vspace{2pt}


\begin{flushleft}
Brown, A. F., Ettner, S. L., Piette, J., Weinberger, M., Gregg, E., Shapiro, M. F., Karter, A. J., Safford, M., et al. (2004). `Socioeconomic position and health among persons with diabetes mellitus: a conceptual framework and review of the literature'. \textit{Epidemiologic reviews}, 26(1), pp.63-77.
\end{flushleft}
\vspace{2pt}


\begin{flushleft}
Centers for Disease Control and Prevention (2024). \textit{Diabetes Risk Factors}. Available at: \url{https://www.cdc.gov/diabetes/risk-factors/index.html} (Accessed: 10 July 2024).
\end{flushleft}
\vspace{2pt}


\begin{flushleft}
Cuadros, D. F., Li, J., Musuka, G., and Awad, S. F. (2021). `Spatial epidemiology of diabetes: methods and insights'. \textit{World journal of diabetes}, 12(7), p.1042.
\end{flushleft}
\vspace{2pt}


\begin{flushleft}
Dagogo-Jack, S. (ed.) (2017). \textit{Diabetes mellitus in developing countries and underserved communities}. Springer International Publishing.
\end{flushleft}
\vspace{2pt}


\begin{flushleft}
Daneman, D. (2006). `Type 1 diabetes'. \textit{The Lancet}, 367(9513), pp.847-858.
\end{flushleft}
\vspace{2pt}


\begin{flushleft}
Dankwa-Mullan, I., Rhee, K. B., Williams, K., Sanchez, I., Sy, F. S., Stinson Jr, N., and Ruffin, J. (2010). `The science of eliminating health disparities: summary and analysis of the NIH summit recommendations'. \textit{American journal of public health}, 100(S1), S12-S18.
\end{flushleft}
\vspace{2pt}


\begin{flushleft}
de Wit, M., Trief, P. M., Huber, J. W., and Willaing, I. (2020). `State of the art: understanding and integration of the social context in diabetes care'. \textit{Diabetic Medicine}, 37(3), pp.473-482.
\end{flushleft}
\vspace{2pt}


\begin{flushleft}
Elliott, P., and Wartenberg, D. (2004). `Spatial epidemiology: current approaches and future challenges'. \textit{Environmental health perspectives}, 112(9), pp.998-1006.
\end{flushleft}
\vspace{2pt}


\begin{flushleft}
Epifano, L., Di Vincenzo, A., Fanelli, C., Porcellati, E., Perriello, G., De Feo, P., Motolese, M., Brunetti, P., et al. (1992). `Effect of cigarette smoking and of a transdermal nicotine delivery system on glucoregulation in type 2 diabetes mellitus'. \textit{European journal of clinical pharmacology}, 43, pp.257-263.
\end{flushleft}
\vspace{2pt}


\begin{flushleft}
Ford, E., Boyd, A., Bowles, J. K., Havard, A., Aldridge, R. W., Curcin, V., Greiver, M., Harron, K., et al. (2019). `Our data, our society, our health: A vision for inclusive and transparent health data science in the United Kingdom and beyond'. \textit{Learning health systems}, 3(3), e10191.
\end{flushleft}
\vspace{7pt}


\begin{flushleft}
Frati, A. C., Iniestra, F., and Ariza, C. R. (1996). `Acute effect of cigarette smoking on glucose tolerance and other cardiovascular risk factors'. \textit{Diabetes care}, 19(2), pp.112-118.
\end{flushleft}
\vspace{2pt}


\begin{flushleft}
Gebreab, S. Y., Hickson, D. A., Sims, M., Wyatt, S. B., Davis, S. K., Correa, A., and Diez-Roux, A. V. (2017). `Neighborhood social and physical environments and type 2 diabetes mellitus in African Americans: The Jackson Heart Study'. \textit{Health \& place}, 43, pp.128-137.
\end{flushleft}
\vspace{2pt}


\begin{flushleft}
Goldstein, B. J. (2002). `Insulin resistance as the core defect in type 2 diabetes mellitus'. \textit{The American journal of cardiology}, 90(5), pp.3-10.
\end{flushleft}
\vspace{2pt}


\begin{flushleft}
González, E. M., Johansson, S., Wallander, M. A., and Rodríguez, L. G. (2009). `Trends in the prevalence and incidence of diabetes in the UK: 1996–2005'. \textit{Journal of Epidemiology \& Community Health}, 63(4), pp.332-336.
\end{flushleft}
\vspace{2pt}


\begin{flushleft}
Henning, R. J. (2018). `Type-2 diabetes mellitus and cardiovascular disease'. \textit{Future cardiology}, 14(6), pp.491-509.
\end{flushleft}
\vspace{2pt}


\begin{flushleft}
Hex, N., MacDonald, R., Pocock, J., Uzdzinska, B., Taylor, M., Atkin, M., Wild, S., Beba, H., et al. (2024). `Estimation of the direct health and indirect societal costs of diabetes in the UK using a cost of illness model'. \textit{Diabetic Medicine}, e15326.
\end{flushleft}
\vspace{2pt}


\begin{flushleft}
Hill-Briggs, F., Adler, N. E., Berkowitz, S. A., Chin, M. H., Gary-Webb, T. L., Navas-Acien, A., Thornton, P. L. and Haire-Joshu, D. (2021). `Social determinants of health and diabetes: a scientific review'. \textit{Diabetes care}, 44(1), p.258.
\end{flushleft}
\vspace{2pt}


\begin{flushleft}
InterAct Consortium robert. (2013). `The link between family history and risk of type 2 diabetes is not explained by anthropometric, lifestyle or genetic risk factors: the EPIC-InterAct study', \textit{Diabetologia}, 56, pp.60-69.
\end{flushleft}
\vspace{2pt}


\begin{flushleft}
James, J. (2015). \textit{The health of populations: Beyond medicine}. Academic Press.
\end{flushleft}
\vspace{2pt}


\begin{flushleft}
Jonsson, B. (1998). `The economic impact of diabetes'. \textit{Diabetes care}, 21(Supplement\_3), C7-C10.
\end{flushleft}
\vspace{2pt}


\begin{flushleft}
Klein, S., Gastaldelli, A., Yki-Järvinen, H., and Scherer, P. E. (2022). `Why does obesity cause diabetes?'. \textit{Cell metabolism}, 34(1), pp.11-20.
\end{flushleft}
\vspace{2pt}


\begin{flushleft}
Knott, C., Bell, S., and Britton, A. (2015). `Alcohol consumption and the risk of type 2 diabetes: a systematic review and dose-response meta-analysis of more than 1.9 million individuals from 38 observational studies'. \textit{Diabetes care}, 38(9), pp.1804-1812.
\end{flushleft}
\vspace{2pt}


\begin{flushleft}
Lang, S. J., Abel, G. A., Mant, J., and Mullis, R. (2016). `Impact of socioeconomic deprivation on screening for cardiovascular disease risk in a primary prevention population: a cross-sectional study'. \textit{BMJ open}, 6(3), e009984.
\end{flushleft}
\vspace{2pt}


\begin{flushleft}
Levene, L. S., Baker, R., Bankart, J., Walker, N., and Wilson, A. (2019). `Socioeconomic deprivation scores as predictors of variations in NHS practice payments: a longitudinal study of English general practices 2013–2017'. \textit{British Journal of General Practice}, 69(685), e546-e554.
\end{flushleft}
\vspace{2pt}


\begin{flushleft}
Lewer, D., Aldridge, R. W., Menezes, D., Sawyer, C., Zaninotto, P., Dedicoat, M., Ahmed, I., Luchenski, S., et al. (2019). `Health-related quality of life and prevalence of six chronic diseases in homeless and housed people: a cross-sectional study in London and Birmingham, England'. \textit{BMJ open}, 9(4), e025192.
\end{flushleft}
\vspace{2pt}


\begin{flushleft}
Maier, W. (2017). `Indices of multiple deprivation for the analysis of regional health disparities in Germany: experiences from epidemiology and healthcare research'. \textit{Bundesgesundheitsblatt-Gesundheitsforschung-Gesundheitsschutz}, 60, pp.1403-1412.
\end{flushleft}
\vspace{2pt}


\begin{flushleft}
Malik, V. S., Popkin, B. M., Bray, G. A., Després, J. P., Willett, W. C., and Hu, F. B. (2010). `Sugar-sweetened beverages and risk of metabolic syndrome and type 2 diabetes: a meta-analysis'. \textit{Diabetes care}, 33(11), pp.2477-2483.
\end{flushleft}
\vspace{2pt}


\begin{flushleft}
Marill, T., and Green, D. (1963). `On the effectiveness of receptors in recognition systems'. \textit{IEEE transactions on Information Theory}, 9 (1), pp.11-17.
\end{flushleft}
\vspace{2pt}


\begin{flushleft}
Marks, J. B., and Raskin, P. (2000). `Cardiovascular risk in diabetes: a brief review'. \textit{Journal of Diabetes and its Complications}, 14(2), pp.108-115.
\end{flushleft}
\vspace{2pt}


\begin{flushleft}
Marmot, M., and Allen, J. J. (2014). `Social determinants of health equity'. \textit{American journal of public health}, 104(S4), S517-S519.
\end{flushleft}
\vspace{2pt}


\begin{flushleft}
McLafferty, S. L. (2003). `GIS and health care'. \textit{Annual review of public health}, 24(1), pp.25-42.
\end{flushleft}
\vspace{2pt}


\begin{flushleft}
McLeod, K. S. (2000). `Our sense of Snow: the myth of John Snow in medical geography'. \textit{Social science \& medicine}, 50(7-8), pp.923-935.
\end{flushleft}
\vspace{2pt}


\begin{flushleft}
Ministry of Housing, Communities and Local Government (2019). \textit{English indices of deprivation 2019}. Available at: \url{https://www.gov.uk/government/statistics/english-indices-of-deprivation-2019} (Accessed: 21 May 2024).
\end{flushleft}
\vspace{2pt}


\begin{flushleft}
Ministry of Housing, Communities and Local Government (2019). \textit{The English Indices of Deprivation 2019 Research report}. Available at: \url{https://assets.publishing.service.gov.uk/media/5d8b364ced915d03709e3cf2/IoD2019_Research_Report.pdf} (Accessed: 15 June 2024).
\end{flushleft}
\vspace{2pt}


\begin{flushleft}
Ministry of Housing, Communities and Local Government (2019). \textit{The English Indices of Deprivation 2019 (IoD2019) Statistical Release}. Available at: \url{https://assets.publishing.service.gov.uk/media/5d8e26f6ed915d5570c6cc55/IoD2019_Statistical_Release.pdf} (Accessed: 15 June 2024).
\end{flushleft}
\vspace{2pt}


\begin{flushleft}
Moore, D. A., and Carpenter, T. E. (1999). `Spatial analytical methods and geographic information systems: use in health research and epidemiology'. \textit{Epidemiologic reviews}, 21(2), pp.143-161.
\end{flushleft}
\vspace{2pt}


\begin{flushleft}
Muir, K., Rattanamongkolgul, S., Smallman-Raynor, M., Thomas, M., Downer, S., and Jenkinson, C. (2004). `Breast cancer incidence and its possible spatial association with pesticide application in two counties of England'. \textit{Public health}, 118(7), pp.513-520.
\end{flushleft}
\vspace{2pt}


\begin{flushleft}
Myerson, R., Lu, T., Peters, A., Fox, S., and Huang, E. (2020). `Impact of health insurance policy on diabetes management'. \textit{Behavioral Diabetes: Social Ecological Perspectives for Pediatric and Adult Populations}, pp.491-504.
\end{flushleft}
\vspace{2pt}


\begin{flushleft}
Nazarko, L. (2023). `Type 2 diabetes: An overview of risk factors and prevention of onset'. \textit{Nursing Times}, 2.
\end{flushleft}
\vspace{2pt}


\begin{flushleft}
NHS England (2023). \textit{National Diabetes Audit 2021-22, Report 1: Care Processes and Treatment Targets, Detailed Analysis Report}. Available at: \url{https://digital.nhs.uk/data-and-information/publications/statistical/national-diabetes-audit/report-1-care-processes-and-treatment-targets-2021-22-full-report} (Accessed: 15 June 2024).
\end{flushleft}
\vspace{2pt}


\begin{flushleft}
NHS England (2022). \textit{NHS prevention programme cuts chances of type 2 diabetes for thousands}. Available at: \url{https://www.england.nhs.uk/2022/03/nhs-prevention-programme-cuts-chances-of-type-2-diabetes-for-thousands/} (Accessed: 15 June 2024).
\end{flushleft}
\vspace{2pt}


\begin{flushleft}
NHS England. (2020). \textit{Patients Registered at a GP Practice January 2020; Special Topic}. Available at: \url{https://digital.nhs.uk/data-and-information/publications/statistical/patients-registered-at-a-gp-practice/january-2020} (Accessed: 14 June 2024).
\end{flushleft}
\vspace{7pt}


\begin{flushleft}
NHS England (2024). \textit{Quality and Outcomes Framework}. Available at: \url{https://qof.digital.nhs.uk} (Accessed: 21 July 2024).
\end{flushleft}
\vspace{7pt}


\begin{flushleft}
NHS England (2023). \textit{Quality and Outcomes Framework, 2022-23}. Available at: \url{https://digital.nhs.uk/data-and-information/publications/statistical/quality-and-outcomes-framework-achievement-prevalence-and-exceptions-data/2022-23} (Accessed: 21 July 2024).
\end{flushleft}
\vspace{7pt}


\begin{flushleft}
NHS England (2022). \textit{Update on Quality Outcomes Framework changes for 2022/23}
\url{https://www.england.nhs.uk/wp-content/uploads/2022/03/B1333_Update-on-Quality-Outcomes-Framework-changes-for-2022-23_310322.pdf} (Accessed: 21 July 2024).
\end{flushleft}
\vspace{7pt}


\begin{flushleft}
Nykiforuk, C. I., and Flaman, L. M. (2011). `Geographic information systems (GIS) for health promotion and public health: a review'. \textit{Health promotion practice}, 12(1), pp.63-73.
\end{flushleft}
\vspace{2pt}


\begin{flushleft}
Office for Health Improvement and Disparities (2023). \textit{Child and maternal health statistics}. Available at: \url{https://www.gov.uk/government/collections/child-and-maternal-health-statistics#overview-of-child-health-and-child-health-profiles} (Accessed: 14 July 2024).
\end{flushleft}
\vspace{2pt}


\begin{flushleft}
Office for Health Improvement and Disparities (2017). \textit{Local Alcohol Profiles for England (LAPE)}. Available at: \url{https://www.gov.uk/government/collections/local-alcohol-profiles-for-england-lape} (Accessed: 14 July 2024).
\end{flushleft}
\vspace{2pt}


\begin{flushleft}
Office for Health Improvement and Disparities (no date). \textit{Statistics on OHID}. Available at: \url{https://www.gov.uk/government/organisations/office-for-health-improvement-and-disparities/about/statistics} (Accessed: 20 June 2024).
\end{flushleft}
\vspace{2pt}


\begin{flushleft}
Office for National Statistics. (2021). \textit{Census}. Available at: \url{https://www.ons.gov.uk/census} (Accessed: 1 June 2024).
\end{flushleft}
\vspace{7pt}


\begin{flushleft}
Office for National Statistics. (2024). \textit{Middle layer Super Output Area population estimates}. Available at: \url{https://www.ons.gov.uk/peoplepopulationandcommunity/populationandmigration/populationestimates/datasets/middlesuperoutputareamidyearpopulationestimates} (Accessed: 1 June 2024).
\end{flushleft}
\vspace{7pt}


\begin{flushleft}
Office for National Statistics (2024). \textit{MSOA (2011) to MSOA (2021) to Local Authority District (2022) Exact Fit Lookup for EW (V2)}. Available at: \url{https://www.data.gov.uk/dataset/da36cac8-51c4-4d68-a4a9-37ac47d2a4ba/msoa-2011-to-msoa-2021-to-local-authority-district-2022-exact-fit-lookup-for-ew-v2} (Accessed: 20 June 2024).
\end{flushleft}
\vspace{2pt}


\begin{flushleft}
Office for National Statistics (2021). \textit{Output Area (2011) to LSOA (2011) to MSOA (2011) LAD to LEP (April 2021) Best Fit Lookup in EN (V3)}. Available at: \url{https://geoportal.statistics.gov.uk/documents/698bb0f385914f79bc68298700acdba8/about} (Accessed: 20 June 2024).
\end{flushleft}
\vspace{2pt}


\begin{flushleft}
Office for National Statistics (no date). \textit{Statistical geographies}. Available at: \url{https://www.ons.gov.uk/methodology/geography/ukgeographies/statisticalgeographies} (Accessed: 1 June 2024).
\end{flushleft}
\vspace{2pt}


\begin{flushleft}
Osayomi, T. (2019). `The emergence of a diabetes pocket in Nigeria: The result of a spatial analysis'. \textit{GeoJournal}, 84, pp.1149-1164.
\end{flushleft}
\vspace{2pt}


\begin{flushleft}
Ozougwu, J. C., Obimba, K. C., Belonwu, C. D., and Unakalamba, C. B. (2013). `The pathogenesis and pathophysiology of type 1 and type 2 diabetes mellitus'. \textit{J Physiol Pathophysiol}, 4(4), pp.46-57.
\end{flushleft}
\vspace{2pt}


\begin{flushleft}
Patnaik, C. and Dutta, A. (2023). `Green finance and economic efficiency in India and regional disparity: An inquiry into its influence using spatial data analysis'. \textit{Journal of Advanced Zoology}, 44, pp. 4169-4179.
\end{flushleft}
\vspace{2pt}


\begin{flushleft}
Payne, R. A., and Abel, G. A. (2012). `UK indices of multiple deprivation-a way to make comparisons across constituent countries easier'. \textit{Health Stat Q}, 53(22), pp.2015-2016.
\end{flushleft}
\vspace{2pt}


\begin{flushleft}
Pettitt, D. J., Talton, J., Dabelea, D., Divers, J., Imperatore, G., Lawrence, J. M., Liese, A. D., Linder, B., et al. (2014). `Prevalence of diabetes in US youth in 2009: the SEARCH for diabetes in youth study'. \textit{Diabetes care}, 37(2), pp.402-408.
\end{flushleft}
\vspace{2pt}


\begin{flushleft}
Pham, T. M., Carpenter, J. R., Morris, T. P., Sharma, M., and Petersen, I. (2019). `Ethnic differences in the prevalence of type 2 diabetes diagnoses in the UK: cross-sectional analysis of the health improvement network primary care database'. \textit{Clinical epidemiology}, pp.1081-1088.
\end{flushleft}
\vspace{2pt}


\begin{flushleft}
Public Health England (2014). \textit{Adult obesity and type 2 diabetes}. Available at: \url{https://assets.publishing.service.gov.uk/media/5a7f069140f0b6230268d059/Adult_obesity_and_type_2_diabetes_.pdf} (Accessed: 21 June 2024).
\end{flushleft}
\vspace{2pt}


\begin{flushleft}
Public Health England (2016). \textit{Diabetes prevalence model: Briefing paper}. Available at: \url{https://assets.publishing.service.gov.uk/media/5a82c07340f0b6230269c82d/Diabetesprevalencemodelbriefing.pdf} (Accessed: 21 June 2024).
\end{flushleft}
\vspace{2pt}


\begin{flushleft}
Qi, X., Jia, Y., Pan, C., Li, C., Wen, Y., Hao, J., and Zhang, F. (2022). `Index of multiple deprivation contributed to common psychiatric disorders: a systematic review and comprehensive analysis'. \textit{Neuroscience \& Biobehavioral Reviews}, 140, 104806.
\end{flushleft}
\vspace{2pt}


\begin{flushleft}
Rajpal, A., Rahimi, L., and Ismail‐Beigi, F. (2020). `Factors leading to high morbidity and mortality of COVID‐19 in patients with type 2 diabetes'. \textit{Journal of diabetes}, 12(12), pp.895-908.
\end{flushleft}
\vspace{2pt}


\begin{flushleft}
Ramachandran, A. (2014). `Know the signs and symptoms of diabetes'. \textit{Indian Journal of Medical Research}, 140(5), pp.579-581.
\end{flushleft}
\vspace{2pt}


\begin{flushleft}
Robinson, W. S. (1950). `Ecological correlations and the behavior of individuals'. \textit{International journal of epidemiology}, 38(2), pp.337-341.
\end{flushleft}
\vspace{2pt}


\begin{flushleft}
Rose, G. (2001). `Sick individuals and sick populations'. \textit{International journal of epidemiology}, 30(3), pp.427-432.
\end{flushleft}
\vspace{2pt}


\begin{flushleft}
Shang, M., Zhang, S. and Yang, Q. (2024). `The spatial role and influencing mechanism of the digital economy in empowering high-quality economic development'. \textit{Sustainability}, 16, p. 1425.
\end{flushleft}
\vspace{2pt}


\begin{flushleft}
Smith, D. M., Vogel, C., Campbell, M., Alwan, N. and Moon, G. (2021). `Adult diet in England: Where is more support needed to achieve dietary recommendations?'. \textit{Plos one}, 16(6), e0252877.
\end{flushleft}
\vspace{7pt}


\begin{flushleft}
Solar, O., and Irwin, A. (2010). \textit{A conceptual framework for action on the social determinants of health}. WHO Document Production Services.
\end{flushleft}
\vspace{2pt}


\begin{flushleft}
Soljak, M., Samarasundera, E., Indulkar, T., Walford, H., and Majeed, A. (2011). `Variations in cardiovascular disease under-diagnosis in England: national cross-sectional spatial analysis'. \textit{BMC cardiovascular disorders}, 11, pp.1-11.
\end{flushleft}
\vspace{2pt}


\begin{flushleft}
Strong, M., Maheswaran, R., Pearson, T., and Fryers, P. (2007). `A method for modelling GP practice level deprivation scores using GIS'. \textit{International Journal of Health Geographics}, 6, pp.1-11.
\end{flushleft}
\vspace{2pt}


\begin{flushleft}
Strong, M., Maheswaran, R., and Radford, J. (2006). `Socioeconomic deprivation, coronary heart disease prevalence and quality of care: a practice-level analysis in Rotherham using data from the new UK general practitioner Quality and Outcomes Framework'. \textit{Journal of Public Health}, 28(1), pp.39-42.
\end{flushleft}
\vspace{2pt}


\begin{flushleft}
Sun, Y., Hu, X., Huang, Y., and On Chan, T. (2020). `Spatial patterns of childhood obesity prevalence in relation to socioeconomic factors across England'. \textit{ISPRS International Journal of Geo-Information}, 9(10), p.599.
\end{flushleft}
\vspace{2pt}


\begin{flushleft}
Sun, Y., Hu, X., and Xie, J. (2021). `Spatial inequalities of COVID-19 mortality rate in relation to socioeconomic and environmental factors across England'. \textit{Science of the total environment}, 758, 143595.
\end{flushleft}
\vspace{2pt}


\begin{flushleft}
Tene, O. and Polonetsky, J. (2012). `Big data for all: Privacy and user control in the age of analytics'. \textit{Nw. J. Tech. \& Intell. Prop.}, 11, 239.
\end{flushleft}
\vspace{7pt}


\begin{flushleft}
The Scottish Public Health Observatory (2023). \textit{Risk factors: overview}. Available at: \url{https://www.scotpho.org.uk/risk-factors/} (Accessed: 25 June 2024).
\end{flushleft}
\vspace{2pt}


\begin{flushleft}
Tonstad, S., Butler, T., Yan, R., and Fraser, G. E. (2009). `Type of vegetarian diet, body weight, and prevalence of type 2 diabetes'. \textit{Diabetes care}, 32(5), pp.791-796.
\end{flushleft}
\vspace{2pt}


\begin{flushleft}
Turi, K. N., and Grigsby-Toussaint, D. S. (2017). `Spatial spillover and the socio-ecological determinants of diabetes-related mortality across US counties'. \textit{Applied geography}, 85, pp.62-72.
\end{flushleft}
\vspace{2pt}


\begin{flushleft}
Willi, C., Bodenmann, P., Ghali, W. A., Faris, P. D., and Cornuz, J. (2007). `Active smoking and the risk of type 2 diabetes: a systematic review and meta-analysis'. \textit{Jama}, 298(22), pp.2654-2664.
\end{flushleft}
\vspace{2pt}


\begin{flushleft}
Whitehouse, S. (2018). \textit{Estimating the diagnosis of health conditions}. Available at: \url{http://dataunlocked.co.uk/estimating-the-diagnosis-of-health-conditions/?doing_wp_cron=1719750389.6598451137542724609375} (Accessed: 7 July 2024).
\end{flushleft}
\vspace{2pt}


\begin{flushleft}
WHO Commission on Social Determinants of Health, and World Health Organization. (2008). \textit{Closing the gap in a generation: health equity through action on the social determinants of health: Commission on Social Determinants of Health final report}. World Health Organization.
\end{flushleft}
\vspace{2pt}


\begin{flushleft}
World Health Organization (2023). \textit{Diabetes}. Available at: \url{https://www.who.int/news-room/fact-sheets/detail/diabetes} (Accessed: 14 June 2024).
\end{flushleft}
\vspace{2pt}


\begin{flushleft}
World Health Organization (no date). \textit{Health Inequality Monitor}. Available at: \url{https://www.who.int/data/inequality-monitor} (Accessed: 20 June 2024).
\end{flushleft}
\vspace{2pt}


\begin{flushleft}
World Health Organization (no date). \textit{World Diabetes Day}. Available at: \url{https://www.who.int/campaigns/world-diabetes-day} (Accessed: 14 June 2024).
\end{flushleft}
\vspace{2pt}


\begin{flushleft}
You, H., Zhou, D., Wu, S., Hu, X., and Bie, C. (2020). `Social deprivation and rural public health in China: exploring the relationship using spatial regression'. \textit{Social indicators research}, 147, pp.843-864.
\end{flushleft}
\vspace{2pt}







\end{Reference}












