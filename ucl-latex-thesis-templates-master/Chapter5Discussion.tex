\chapter{Discussion}
\label{chap:5}

\section{General Discussion}
\label{chap:5.1}
Through initial OLS regression analysis, we found that the Asian population proportion, social deprivation (IMD decile) and obesity prevalence significantly influenced diabetes prevalence. However, significant spatial autocorrelation in the OLS model residuals indicated potential spatial dependency among variables. Thus, we ultimately chose the SDM to better explain these spatial effects.

The SDM results showed that certain variables not only directly impacted diabetes prevalence but also exhibited spatial spillover effects, particularly the Asian population, elderly population proportion and social deprivation. This suggests that population structure and socioeconomic conditions in neighboring areas may significantly affect diabetes prevalence in the current area. Specifically, the research found that when neighboring areas had a higher Asian population proportion, the diabetes prevalence in the current area also increased correspondingly. The elderly population also demonstrated this spatial effect. This may reflect that the area clustering effect of lifestyle, community characteristics and cultural habits of related groups on health status. In England, Asian communities typically have strong cohesion and are concentrated in some large cities. Asian restaurants tend to open in these cohesive communities, further attracting more Asian populations to settle. These restaurants are often concentrated in large cities or around universities, reinforcing the clustering effect of Asian populations in these areas. Moreover, traditional Asian dietary habits are typically rich in carbohydrates, such as rice and noodles. The provision of this diet by restaurants not only meets cultural demands but also, to some extent, promotes the continuation of these dietary habits, potentially increasing the risk of diabetes.

In contrast, whilst the Black population was also significant, it did not exhibit apparent spatial spillover effects, suggesting that the distribution of this group may have a more localized impact on diabetes prevalence without forming widespread spatial effects. Although obesity prevalence showed significant direct effects in the model, it did not demonstrate spatial spillover effects. This implies that the impact of obesity on diabetes prevalence is more confined to specific areas rather than through effects transmitted across neighboring areas. Furthermore, the spillover effect of social deprivation level suggests that economically disadvantaged areas not only have higher diabetes prevalence themselves but their deprived conditions also affect surrounding areas, exacerbating health inequalities in diabetes across different regions. The spillover effect of socioeconomic factors also appeared in Turi's study on DM in the United States (2017), indicating that socioeconomic inequality is an important determinant in many study areas.

The spatial spillover effect of IMD decile suggests that reducing social deprivation is also crucial for diabetes control. Policymakers should strive to reduce socioeconomic inequalities by improving education and employment opportunities, as well as housing and healthcare resources, to lower the risk of diabetes. These measures can not only directly aid deprived areas but also help alleviate pressure on neighboring areas. Regarding obesity, although it did not show apparent spatial spillover effects, its impact in local areas cannot be ignored. Public health strategies should focus on improving people's healthy dietary habits to reduce its direct impact on diabetes. To mitigate the impact of population spillover effects on diabetes prevalence, I suggest promoting health education and preventive measures through institutions such as universities and companies. Universities and companies possess extensive social networks and influence, enabling effective dissemination of knowledge about healthy lifestyles, especially in areas with concentrated high-risk groups.


\section{Limitations}
\label{chap:5.2}
This study also has several limitations. Firstly, there is inconsistency in the temporal alignment of the datasets. For instance, whilst we analyzed diabetes prevalence for 2022/23, the adult dietary data used as an independent variable was modeled based on survey results from 2008-2016 (Smith et al., 2021), potentially raising issues of data obsolescence. Moreover, although we used the 2019 IMD index, which is relatively close to 2022, some of the underlying indicators for IMD were collected at inconsistent and sometimes distant time points. For example, the income indicator is based on 2015 data, which may impact the analysis.
Secondly, although we treated IMD decile as a continuous variable based on literature, to further verify the robustness of results, sensitivity analysis could be conducted, such as coding IMD decile as dummy variables for comparative experiments (Lang et al., 2016).
Thirdly, Soljak et al. (2011) noted that whilst QOF data accurately reflect the prevalence of diagnosed diseases, undiagnosed cases remain a consideration. Future research could incorporate more comprehensive disease prevalence models, such as those provided by the Association of Public Health Observatories, to compare and validate QOF register data, thereby improving the accuracy of diabetes prevalence estimates.
Furthermore, as this study uses ecological data, results cannot be directly applied to individual-level interpretations. This may lead to ecological fallacy, where group-level associations are misinterpreted as individual-level causal relationships, resulting in erroneous inferences (Robinson, 1950). A more reasonable approach would be to use case-control or cohort studies for verification (Elliott and Wartenberg, 2004).
Lastly, cross-sectional studies have limitations in exploring causal relationships. Future research could develop spatiotemporal models to capture more refined spatial distribution and change processes of diabetes, thereby gaining a deeper understanding of the causal relationships between influencing factors and diabetes.