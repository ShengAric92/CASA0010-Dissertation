\chapter{Literature Review}
\label{chap:2}

This chapter will provide a systematic and critical review of the existing literature on DM, and it contains five sections. First, we will review the research on DM risk factors and explore the non-modifiable and modifiable factors. Then the influence of SDOH on DM will be reviewed and discussed. The third section will focus on the application of spatial epidemiology and show its contribution to understanding health inequalities in DM. The fourth section will introduce two important tools often used in public health research in England and their applications. Finally, the research gaps in the existing literature will be discussed.

\section{Risk Factors for DM}
\label{sec:2.1}
A risk factor for a disease is a factor that increases the likelihood of developing the disease or the severity of its progression (Borgnakke, 2016). To formulate reasonable and effective preventive measures, gaining an in-depth understanding of the risk factors contributing to DM is crucial. When discussing risk factors for diseases, they are typically divided into two categories: non-modifiable (cannot be changed) and modifiable (can be altered) risk factors (Nazarko, 2023).

Non-modifiable risk factors for DM generally include ethnicity, sex, advancing age, and family history. For example, numerous studies have indicated that the prevalence of DM varies across ethnic groups. Pham et al. (2019) studied 400,000 individuals in a UK primary care setting. They found that the prevalence of T2DM was significantly higher among Asian and Black ethnic groups compared to White groups. Pettitt et al. (2014) revealed that White ethnic groups exhibit a slightly higher prevalence of T1DM. González et al. (2009) conducted a study on the ten-year trend of increasing DM prevalence in the UK, clarifying that it rises linearly with advancing age. This is further emphasized by a DM prevalence model published by the Public Health England (2016), which shows that the prevalence of diabetes is 9.0\% in the 45-54 age group, rising to 23.8\% in those aged 75 and above. The model also highlights differences between sexes, with a slightly higher prevalence of 9.6\% in men compared to 7.6\% in women. InterAct Consortium (2013) noted that individuals with a first-degree relative with T2DM and those with both parents affected by T2DM had two to three times and five to six times the prevalence, respectively, compared to individuals without a family history of T2DM.

The discussion of modifiable risk factors for DM in the literature is relatively complex. According to Boehme, Esenwa and Elkind (2017), modifiable risk factors can be further divided into two types: behavioral risk factors and clinical risk factors. Behavioral risk factors mainly reflect an individual's lifestyle and actions, while clinical risk factors are physiological attributes that increase the risk of disease and require clinical assessment through measurement or biochemical analysis (The Scottish Public Health Observatory, 2023). James (2015), in his book, emphasized that behavioral risk factors are the primary factors, while the latter is the secondary factors. He noted that reducing exposure to the primary factors often serves as a strategy to prevent the development of the secondary factors.

Since T1DM is triggered by an autoimmune reaction, there is currently no effective prevention method. However, T2DM is closely linked to lifestyle and can be prevented and managed through lifestyle interventions (Centers for Disease Control and Prevention, 2024). Behavioral risk factors for T2DM typically include smoking, excessive alcohol consumption, and unhealthy dietary habits. Research indicates that smoking not only impairs insulin sensitivity (Frati, Iniestra and Ariza, 1996), but nicotine, the addictive substance in tobacco, has also been shown to alter glucose homeostasis (Epifano et al., 1992). Willi et al. (2007), in a systematic review, found that active smoking significantly increases the risk of T2DM. Knott's study (2015) demonstrated that alcohol consumption as a risk factor has a threshold; while moderate drinking may even reduce the risk of T2DM, excessive drinking increases the risk. Tonstad's research (2009) indicated that the prevalence of T2DM is significantly lower in vegans than in non-vegans. Additionally, Malik's study (2010) found that high consumption of sugar-sweetened beverages (SSBs) not only leads to weight gain but is also closely linked to the development of T2DM.

Obesity has been widely recognized in the literature as one of the most important clinical risk factors for DM. Klein's study (2022) pathologically explained the link between obesity and T2DM, noting that fat accumulation causes functional changes in beta cells and leads to insulin resistance in multiple organs. A report by Public Health England (2014) specifically pointed out that obese adults in England are five times more likely to develop diabetes compared to adults of normal weight, with approximately 90\% of adult T2DM patients being either overweight or obese. In addition, clinical risk factors for DM also include other medical conditions such as hypertension, hyperlipidaemia, and cardiovascular disease (Borgnakke, 2016). Some studies also highlight that DM is itself a risk factor for cardiovascular disease, with the two conditions mutually influencing each other (Marks and Raskin, 2000).


\section{Social Determinants of Health and DM}
\label{sec:2.2}
The above literature reviewed many risk factors for DM, and most of this literature focused on individual-level studies, with the risk factors primarily related to individual biological characteristics and living habits. Although it is important to understand risk factors at the individual level, this often overlooks the social and environmental context of the patient, which may also influence the prevalence and incidence of DM through complex mechanisms. For this reason, the research paradigm of epidemiology has begun to pay increasing attention to population health outcomes, emphasizing the study of population characteristics and their influencing factors. Rose criticized (2001) the individual-centered approach from an etiological perspective, arguing that focusing on relative risk represents etiological strength but does not measure etiological outcomes. Therefore, she emphasized that identifying the determinants of prevalence and incidence should not be limited to exploring individual characteristics but should focus on population characteristics. Berkman, Kawachi and Glymour (2014) also emphasize, from an intervention perspective, that if we only look for risk factors at the individual level, our interventions will inevitably focus on individual behavior.

Many studies have shown that DM may be more severe in certain groups, and the health disparities that exist make DM more challenging to manage and prevent (Dankwa-Mullan et al., 2010). In this context, the concept of health equity becomes particularly important. Health equity refers to the absence of unfair and avoidable health disparities within population groups due to economic, social, or geographical factors, i.e., all people have a fair opportunity to achieve the highest possible state of health (WHO, no date). Pursuing health equity means valuing everyone equally, striving for the highest possible health standards for all, and working to eliminate health disparities and inequalities (Braveman, 2014). Addressing health equity, therefore, requires a careful understanding of social and environmental factors. Studies have shown that these factors account for 50\% of health outcomes (Marmot and Allen, 2014). These non-medical factors that influence health outcomes are collectively known as social determinants of health (SDOH). SDOH, by definition, refers to people's daily living conditions and the drivers that shape those conditions, and are widely described as the 'causes of the causes' of health problems. Inequalities in social determinants lead to health inequality, which results in unfair and avoidable health disparities in health status (WHO, 2008).

Hill-Briggs et al. (2021) reviewed several common SDOH frameworks, including the WHO Commission on the SDOH (Solar and Irwin, 2010) and the Kaiser Family Foundation SDOH framework (Artiga and Hinton, 2018). They note that there is no single consensus set of factors that defines SDOH, but a common feature among the frameworks is the recognition of socioeconomic determinants as the most important. These frameworks reveal the complex interactions between SDOH factors. To better understand the relationship between SDOH and DM, they proposed and constructed an SDOH framework for DM, where the included SDOH factors were at least reflected in existing SDOH frameworks, and there was sufficient literature to demonstrate that these factors had a significant impact on diabetes. Their framework covers five main aspects: Socioeconomic Status (SES), Neighborhood and Physical Environment, Food Environment, Health Care, and Social Context. The prevalence and incidence of DM showed a gradient correlation at each SES level, with the gradient being steeper at the lower levels (Hill-Briggs et al., 2021). People with lower SES are more likely to develop T2DM and experience more complications than those with higher SES (Brown et al., 2004). Many studies also point out that instability in the built environment and housing is associated with DM. For example, Bilal et al. (2018) pointed out that communities with better walkability have lower prevalence and incidence rates of T2DM. Lewer et al. (2019) examined the prevalence of various chronic diseases in both housed and homeless population in England and found that health disparities between the two groups were significant, with the homeless populations having a much higher prevalence. Myerson et al. (2020), which summarized insurance policies implemented to improve DM prevention and diagnosis, found that expanding insurance coverage for low-income adults may improve DM outcomes. In addition, higher social cohesion (Gebreab et al., 2017) and social support (de Wit et al., 2020) are associated with better quality of life and more effective blood glucose control, which also reduces the prevalence, incidence, and mortality of DM.


\section{Spatial Epidemiology of DM}
\label{sec:2.3}
In the previous sections, we reviewed the main risk factors for DM and explored how epidemiological studies have gradually moved from the individual-level to the population-level research paradigm, revealing the health inequalities in DM and its SDOH framework. However, traditional epidemiology has focused on people and time and less on the geographical and spatial distribution of disease (Moore and Carpenter, 1999), which limits our ability to understand health inequities. To address this issue, spatial epidemiology has gradually become an important research direction, filling the gap in traditional research. Spatial epidemiology focuses on small-area analysis to study and describe the geographical variation of disease concerning socioeconomic, demographic, and behavioral risk factors (Elliott and Wartenberg, 2004). In spatial epidemiology, the concept of place extends beyond individual characteristics to examine individuals' social and environmental contexts and how these interactions influence their health (Cuadros et al., 2021). This aligns with social epidemiology's critique of the notion that all determinants of health must be conceptualized as individual-level attributes.

An early application of spatial epidemiology was disease mapping, with a prime example being the Broad Street cholera outbreak of 1854. In this case, physician John Snow discovered that the source of the cholera infection was contaminated water, not the air. He used a dot map to show that water pumps were at the center of concentrated outbreaks of cholera cases, making a significant contribution to the field of public health (McLeod, 2000). However, early spatial epidemiology faced several challenges, such as the monopoly doctors had over disease data and the difficulty of creating hand-drawn maps (Askari, Gupta and Bengal, 2016), which hindered the technological development of the field and limited its popularity at the time. With the emergence and development of GIS, satellite remote sensing, and computer mapping, along with advancements in quantitative methods, spatial epidemiology has significantly improved and expanded. For example, there is now a large amount of high-resolution social and geographic data available (Nykiforuk and Flaman, 2011). In addition to disease mapping, geographical correlation studies and disease clustering are becoming more common as the development of statistical models makes spatial analysis techniques more powerful. These advancements provide new opportunities and a broader perspective on the causes and prevalence of disease, allowing us to establish a solid geographical foundation for disease prevention and health policy formulation (McLafferty, 2003). For example, Muir et al. (2004) studied the spatial distribution of breast cancer incidence in two counties in England, Lincolnshire, and Leicestershire, and used Moran's I spatial statistical method to determine whether there was spatial autocorrelation in breast cancer cases. They also applied linear regression to analyze the association between pesticide application and breast cancer. Soljak et al. (2011) conducted a cross-sectional analysis to compare the spatial distribution patterns of three types of cardiovascular disease—hypertension, stroke, and coronary heart disease—across England. They used local Moran's I to identify geographic clusters and outliers and applied ordinary least squares (OLS) logistic regression models for fitting.

Many previous studies have used OLS to explore the association between risk factors and health outcomes. However, Anselin and Bera (1998) point out that spatial autocorrelation is crucial in analyzing cross-sectional data in the natural sciences, especially in fields such as epidemiology, ecology, and geology, where data often exhibit spatial autocorrelation. Therefore, OLS models, which assume independence of observations and do not account for spatial autocorrelation, are not always reliable. Recent examples show that spatial regression models have become widespread, with spatial analysis methods being skillfully applied and forming a comprehensive system. For instance, Sun (2021) examined spatial inequalities in COVID-19 mortality in relation to environmental and socioeconomic factors in England. They first used the bivariate Moran's I test to explore the spatial association between COVID-19 mortality and non-COVID-19 mortality. After detecting significant spatial autocorrelation in the residuals of the non-spatial regression model, they applied the eigenvector space filtering (ESF) method and the spatial lag model (SLM). In Sun et al. (2020) studied on the prevalence and socioeconomic factors of childhood obesity in England, global Moran's I and local Moran's I was first used to explore the spatial clustering pattern of prevalence. Then, the ESF method and SLM were used for model fitting and estimation.

Regarding spatial studies of DM, some scholars have pointed out that the current understanding of the geographical analysis and determinants of DM is still limited (Cuadros et al., 2021). Consequently, some researchers have begun to use spatial analysis methods to study DM. Osayomi's (2019) study first explored the spatial distribution pattern of DM prevalence in Nigeria using global Moran's I and local Moran's I, identifying geographic clustering. They then applied a spatial error model (SEM) and hypothesized that sociocultural practices, traditional dietary habits, and lower education levels may be responsible for this geographic pattern of DM prevalence, recommending regional policy interventions. Turi and Grigsby-Toussaint (2017) examined DM-related mortality rates by county in the United States, finding that lower socioeconomic status and lower levels of physical activity characterized areas with high concentrations of DM-related mortality. They also applied the spatial durbin model (SDM) and found that socioeconomic gradients in neighboring counties had spillover effects on DM-related mortality.

After reviewing many applications of spatial epidemiology methods, it is important to recognize that while these developments have led to significant advances and new insights into our understanding of disease, they have also faced criticism and limitations. Elliott and Wartenberg (2004) criticized many research designs in spatial epidemiology for relying on ecological studies, in which the unit of observation is geographical groups, often leading to the ecological fallacy. This term, defined by Robinson (1950), refers to errors that can arise when inferring from group-level data to the individual level, as conclusions drawn at the group level may not always apply to individuals. Elliott and Wartenberg (2004) also emphasize that ecological studies need to be validated at the individual level, for instance, by using case-control studies or cohort studies to avoid ecological fallacies. Other scholars have criticized many studies for favoring more advanced methods rather than addressing the original research question (Albert, Gesler and Levergood, 2000). The essence of epidemiological research is to solve health problems, and it should not fall into the trap of method abuse or technology worship. Unthinkingly pursuing more advanced and complex methods to highlight research results is a case of putting the cart before the horse.


\section{Tools for Public Health Research in England}
\label{sec:2.4}
After reviewing a substantial body of work on applying and critiquing methods in spatial epidemiology research, I’ve noticed that two `tools’ repeatedly feature in most studies conducted in England. These two key `tools’ are the Index of Multiple Deprivation (IMD) and the Quality and Outcomes Framework (QOF). When studying health inequalities across England, researchers can make use of these readily available datasets and indicators, enabling a more thorough understanding and analysis of the relationship between socioeconomic factors and health outcomes. In the following sections, I will provide a detailed overview of IMD and QOF, reviewing their usage in research. Additionally, I will discuss how researchers have expanded on these tools to meet broader research needs.

\subsection{Index of Multiple Deprivation}
\label{sec:2.4.1}
The Index of Multiple Deprivation (IMD) measures relative deprivation in small areas within the UK. The dimensions of the measurement include Income, Education, Employment, Health, and other domains. These components of deprivation are assigned a score according to different weights. The small areas are ranked according to their deprivation score. The development of this index can be applied to a wide range of uses, including resource allocation, policy formulation, and strategic assessment (MHCLG, 2019). It serves as a basis to support researchers and local governments in analyzing resources and addressing inequality issues. The four countries of the UK have their own IMD. However, the geographical units used to calculate the scores, the weights given to each component, and the year and time points at which the data were collected are all different, so they are often not directly comparable (Payne and Abel, 2012).

In England, the earliest IMD, dating back to 2000, was measured at the ward level, while those published in 2004 and later was measured at the Lower layer Super Output Area (LSOA) level. The latest IMD in England was published in 2019. In IMD 2019, deprivation is assessed across seven domains, each with a different weight. Income and Employment are each given the highest weight of 22.5\%. Education and Health each account for 13.5\%, while Crime, Barriers to Housing and Services, and Living Environment each contribute 9.3\%. Each domain has underlying indicators to measure it, and through specific methods, including shrinkage procedures, factor analysis, and standardized sub-domains, the score for each domain is calculated. Finally, these domain scores are combined according to the weight of each domain to obtain the total score (MHCLG, 2019).

The domains of the IMD align closely with the Social Determinants of Health (SDOH) framework, which has led to the rapid rise of IMD use in public health research. Many researchers now utilize the IMD as a variable to measure area-level deprivation, reflecting the overall socio-economic conditions of different regions. For instance, Qi et al. (2022) examined the relationship between social environments, genetic factors, and various mental health disorders. They employed linear and logistic regression models to assess the association between IMD and bipolar disorder, anxiety, and depression, finding that higher IMD scores were significantly associated with increased risk of these conditions.

Other research has explored how area-level IMD scores can be adapted at the GP practice level to accommodate studies where GP practices are the observations (Strong et al., 2007). They critically reviewed several methods for calculating GP-level IMD, including the post-code linked and population-weighted methods. The post-code linked method assigns the IMD score of the area where the GP's post code is located. Conversely, the population-weighted method calculates an IMD score based on the distribution of patients' home addresses across areas. While the former does not account for complex population distributions, the latter may raise ethical concerns as it requires patient post-code information access. The researchers also developed an innovative approach that can accurately assess deprivation at the GP practice level without needing patient address information.

IMD usage has not been confined to the UK; it has become increasingly popular in global public health research. Maier (2017) noted that Germany had developed its own version of the multiple deprivation index, modeled on the UK's approach, and their research revealed a significant link between regional deprivation and both mortality and morbidity. Similarly, You et al. (2020) sought to explain the relationship between social deprivation and public health in rural China. They integrated dozens of indicators, selected a weighting method to assess regional deprivation, and utilized linear regression, spatial lag models, and spatial error models in their analysis.

\subsection{Quality and Outcomes Framework}
\label{sec:2.4.2}
The QOF is an annual incentive and reward programme introduced in 2004, designed to encourage good practice and provide resources. All GP practices in England can voluntarily participate. The QOF assesses performance across five domains: Clinical, Public Health, Public Health (Vaccination and Immunization), Public Health (Additional Services), and Quality Improvement, to evaluate and reflect the overall achievements of a practice (NHS England, 2024). The higher the quality of care a practice provides, the higher its score. NHS England provides annual QOF data through an accessible online database. This transparency helps build trust with patients and the public, while public health researchers and policymakers can easily access the data to support research, foster innovation, and inform policy decisions.

The QOF online database includes data at the GP practice level on the prevalence of various health conditions, achievement, and personalized care adjustments (NHS England, 2023). Epidemiologists have widely used this data in their research. For instance, in their study on the relationship between coronary heart disease and deprivation, Strong, Maheswaran, and Radford (2006) utilized QOF GP practice prevalence data. They also incorporated IMD data and used a population-weighted method to calculate IMD at the GP practice level. Their findings showed a positive correlation between disease prevalence and deprivation but no significant relationship between quality of care indicators and deprivation.

Earlier studies using QOF data mainly focused on GP practice as the observation unit. However, as the demand for area-level research grew, some public health researchers attempted to convert QOF GP practice prevalence data into area-level prevalence estimates. Whitehouse (2018) used population-weighting, based on the distribution of GP practice patients across LSOA, to estimate the prevalence for each LSOA. Since 2020, NHS has provided quarterly updates on the population distribution registered with GP practices (NHS England, 2020), making applying the population-weighting method for converting data across different levels easier. However, Whitehouse (2018) acknowledged that this method is a rough estimate, as it overlooks the complexities of patient mobility and registration patterns. Baker (2021), building on this method, accounted for more complex factors and developed a modeled estimates method, providing more reliable and accurate area-level prevalence data.


\section{Research Gaps}
\label{sec:2.5}
It can be seen from the aforementioned literature review that most existing DM studies are limited to individual-level analysis. Although there are discussions on socioeconomic factors at the group level, these studies tend to overlook the importance of geospatial location and are limited in their ability to analyze health inequalities effectively. While spatial epidemiology has gradually entered the researchers' field of view and has been applied to the study of many diseases, progress in the spatial study of DM remains limited, with the overall application being scarce and the number of studies relatively small. The literature has discussed significant spatial clustering patterns in DM prevalence in countries such as the United States and Nigeria, where socioeconomic factors play a substantial role in the spatial distribution of DM and exhibit spatial effects. However, in studies conducted in England, there are still significant gaps, with insufficient consideration given to the importance of area-level risk factors and spatial analysis in relation to DM prevalence. This research gap has led to an incomplete understanding of certain high-risk areas, affecting the development of targeted public health interventions.

Therefore, this paper aims to conduct an ecological study through spatial analysis of DM prevalence in England, utilizing IMD and QOF datasets to reveal the spatial distribution of various risk factors, including social deprivation, at the MSOA level and their impact on the spatial pattern of DM prevalence. The paper seeks to provide a geographical foundation for more effective public health policies and interventions by filling these research gaps.