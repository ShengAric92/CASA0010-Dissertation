\chapter{Conclusion}
\label{chap:6}

This study identified the significant spatial autocorrelation in DM prevalence in England, indicating a spatial clustering phenomenon in certain areas. Further spatial regression analyses addressed the core research questions and identified several key influencing factors, including Asian population proportions, social deprivation (IMD), and advancing age. Notably, the significant spatial spillover effects of social deprivation and the Asian population suggest that their influence extends beyond specific areas and impacts the surrounding areas through spatial interaction effects. These findings offer an area perspective for future DM prevention and control policies, highlighting the importance of considering not only areas with high prevalence but also the influence of neighboring areas.

The study also found that, although the obesity prevalence was a significant factor in the OLS model, it did not show a spatial spillover effect in the SDM. This may indicate that the effects of obesity are more confined to the individual level rather than spreading through social interactions or neighborhood influences. As a result, future policy interventions may need to prioritize community-level impacts, such as promoting health awareness and healthy lifestyles through public health programs.
Moreover, this study adds a new case to existing spatial epidemiological research, especially in the analysis of noncommunicable diseases. There is considerable potential for further exploration in this area. Future studies could expand the scope of analysis to smaller areas or introduce more complex spatiotemporal dynamic models to track changes in DM prevalence over time.

Despite the valuable insights provided by this study, there are several limitations. First, the ecological nature of the data means that the findings cannot be directly applied to individual-level interpretations. Second, while this study includes a range of socioeconomic and behavioral factors, there may be other potential influences that were not captured. Additionally, while the SDM can reveal spatial spillover effects, cross-sectional studies are limited in their ability to explore causality. Future research could develop spatiotemporal models to capture more detailed spatial distributions and changes in DM prevalence, and better understand the causal relationships between risk factors and DM.
