\chapter{Introduction}
\label{chap:1}

Diabetes mellitus (DM) is a metabolic disease characterized by chronically high blood sugar levels. It has many subtypes, the two most important of which are Type I (T1DM), an autoimmune disease, and Type II (T2DM), a chronic metabolic disorder. Other types of DM include high blood sugar symptoms that occur during pregnancy, known as gestational diabetes, and DM caused by rare factors such as genetic defects, hormonal imbalances, viral infections, or drug exposure (Alam et al., 2014). In patients with T1DM, the immune system destroys the beta cells in the pancreas that secrete insulin, resulting in high blood sugar levels. This type usually develops in childhood and is difficult to prevent, requiring lifelong insulin injection therapy (Daneman, 2006). T2DM, which usually develops in adulthood, is characterized by insufficient insulin production by the patient's own pancreas and a poor response by cells in the body to it, forming what is known as insulin resistance, resulting in excessive glucose in the blood (Goldstein, 2002). It is the most common type of DM, accounting for more than 90\% of all DM cases, compared to about 5\% to 10\% for T1DM (Ozougwu et al., 2013).

Symptoms of DM typically include frequent urination, fatigue, excessive thirst, and unintentional weight loss. DM continually damages various organs in the body and can lead to amputation or death in severe cases. DM is not only a predisposing factor for cardiovascular diseases but also exacerbates them (Henning, 2018), and it increases the risk of complications from other diseases, such as higher mortality in COVID-19 patients with DM (Rajpal, Rahimi and Ismail‐Beigi, 2020). As highlighted by Ramachandran (2014), the effects of hyperglycemia often develop gradually, making individuals less aware of them, and the real damage can occur years before symptoms become apparent. Therefore, although DM is a noncommunicable disease, it has been spreading rapidly yet quietly on a global scale and has drawn significant attention from epidemiologists.


DM has become a global health crisis. In 1991, the International Diabetes Federation (IDF) established World Diabetes Day with the support of the World Health Organization (WHO). It also became an official United Nations Day in 2006 (WHO, no date). According to data published by the WHO (2023), the number of DM patients worldwide increased from 108 million to 422 million between 1980 and 2014. Not only has the number of patients increased dramatically, but the fatality rate has also risen significantly. The fatality rate for DM increased by three percentage points in 2019 compared to 2000, with approximately 2 million people dying from DM and related kidney disease that year. The global deterioration of DM has a severe impact on public health and carries a strong economic burden. More than two decades ago, Jonsson's review (1998) of global DM research revealed that around 2-3\% of countries' total healthcare budgets were consumed by DM. According to a recent review (Dagogo-Jack, 2017), DM accounts for 5-20\% of healthcare budgets in most countries. For the United Kingdom (UK), a recent study estimated the overall cost of DM in 2021/22 at around £14 billion (Hex et al., 2024), including more than £10 billion in direct costs to the NHS, representing 10\% of the NHS's annual budget. To help more people lead healthier lives, the NHS Diabetes Prevention Programme (NHS DPP) (NHS England, 2022) was launched in 2016 to encourage people to prevent T2DM by improving their lifestyles. However, according to a report published by the NHS England (2023), the prevalence of DM in England is not optimistic. It has continued to rise in recent years, from 3.2 million to 3.6 million between 2017 and 2021. Therefore, various evidence shows that research and prevention of DM have become critical issues in public health today, and more research support, policy management, and effective implementation of policies are urgently needed.


To develop effective prevention and management measures, an in-depth understanding of the risk factors leading to DM has been a hot topic for researchers in recent years. Existing research generally divides these risk factors into non-modifiable factors (such as race, gender, age, and family history) and modifiable factors (such as behavioral habits and clinical indicators that can be improved) (Borgnakke, 2016). Although many studies have effectively explained the effects of biological characteristics and lifestyle habits on DM, most focus on the individual level. Many sociologists have also criticized this, and more scholars are beginning to realize that merely looking at the individual may ignore the influence of the broader social and environmental context on DM (Berkman, Kawachi and Glymour, 2014). Some scholars have pointed out the importance of a deep understanding of social determinants of health (SDOH) to address health inequalities in DM and have established the SDOH framework for DM (Hill-Briggs et al., 2021). However, most studies in the literature focus more on people and time but ignore the distribution of diseases in geographical space. With the development of quantitative methods and Geographic Information Systems (GIS), spatial epidemiology has filled this gap and has gradually become more powerful and popular. It could help researchers better understand health inequalities.

Spatial epidemiology used to focus on infectious diseases, but now its application has been extended to the study of noncommunicable diseases (Cuadros et al., 2021). Although some scholars have begun to conduct detailed spatial analyses of DM, for example, Osayomi (2019) explored the spatial distribution pattern of diabetes prevalence in Nigeria, and Turi and Grigsby-Toussaint (2017) used a spatial durbin model (SDM) to reveal a spatial spillover effect on DM-related mortality across counties in the United States, this kind of research remains limited and has yet to receive widespread attention. In addition, there are few spatial studies on DM in England, and most DM studies focus on the individual level. Therefore, by exploring the spatial pattern of DM prevalence in England and attempting to understand and explain the significant risk factors contributing to this pattern, this study hopes to guide preventive measures, contribute to public health, and add fresh research cases to the spatial epidemiology of DM prevalence in England.


This study aims to answer the core research question: \textit{Which risk factors significantly influence the spatial pattern of diabetes mellitus prevalence at the MSOA level in England?} This paper seeks to conduct an ecological study that establishes a spatial regression model by utilizing spatial data on demographics, socioeconomics, living habits, and health care to address this research question. To comprehensively answer this question, the following research objectives will be achieved: 1) To determine whether there is spatial autocorrelation in the DM prevalence in England and whether it shows patterns of clustering or dispersion. 2) To identify which risk factors significantly contribute to the spatial variation in DM prevalence in England. 3) To explore how these factors lead to spatial variation and whether spatial effects are present.

The structure of this paper is as follows: Chapter 2 reviews existing risk factors for DM and critically evaluates the application of spatial epidemiological methods. Chapter 3 details the datasets and methods used in this study. Chapter 4 presents the results of the exploratory spatial data analysis and the findings from the ordinary least squares (OLS) and the spatial regression model. Chapter 5 discusses and interprets the results, offering insights into relevant and feasible DM prevention strategies. Additionally, this chapter addresses the study’s limitations. Chapter 6 summarizes the main contributions of this research and concludes with suggestions for future research prospects.
